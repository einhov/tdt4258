\chapter{Introduction}

When you plug your headphone into your cellphone, opens up Spotify and push play on your favourite song, how do you know there will be any sound? What do really happen? How does your device generate sound? Or even more specific, how do software(Spotify) and hardware(Phone) interact with sound? During this report you will enter this discrete digital and mystical world. In this report we will talk about and discuss how to make sound effects by interacting with different gamepad-buttons using the C programming language. Thus, create software which can communicate with the hardware to generate sound. We will also provide an explanation and insight into the different sound characteristics.

For this exercise, we were supplied with a motherboard, a prototyping board and the GNU C Compiler; GCC with the corresponding documentation, explaining how C files could be linked with the gcc command. We were also supplied with a custom gamepad and manuals for all of these parts. In addition to this, we got a thorough explanation on how to implement interrupts handling and other useful tips for hardware timers in C. 

The exercise specified that we should write a C program that runs directly on the development board without the support of an operating system and which plays different sound effects when different buttons are pressed. Three different sound effects should be made. In order to complete this exercise we needed to understand the the underlying fundamental principles and functions of the sound generator: Digital to Analog Converter (DAC), and studying sound wave synthesis. The EFM32GG reference manual provided was in good stead. 
This leaves us with the following questions; what is a DAC, and how do we generate sound? How energy efficient will this be implemented? And what are the underlying properties of a soundwave?
\chapter{Background and Theory}

The problem is to be solved on the EFM32GG-DK3750 Development Kit from Energy
Micro. The kit contains the EFM32GG990F1024 MCU, a motherboard with various
peripherals, and a prototyping board for connecting external devices.
\cite{DK3750Manual} Further we are also using a custom gamepad made specifically
for the course. To develop and profile the software for the kit we are using the
GNU toolchain and selected tools from the energyAware software suite.

\section{The MCU}

The MCU contains a CPU core from the ARM Cortex-M3 family, 1MB of Flash memory,
128kB of SRAM, a debug interface, an interrupt controller and various
peripherials. Amongst the peripherials are clocks, I/O ports and timers.
For solving this problem we will only need peripherals from
these categories, but the MCU also contains serial interfaces, analog
interfaces, an energy management unit and hardware acceleration for AES
encryption. The Flash memory, SRAM and peripherals are all mapped to the CPU's
system memory map. \cite{EFM32GGManual}

The MCU operates in different energy modes, named EM0 through EM4. At EM0,
called Run Mode, the CPU and all peripherals are active. At EM1, called Sleep
Mode, the CPU is halted in a sleep mode, but all peripherals are still active.
In higher modes, sets of peripherals are deactivated to further lower energy
consumption. Entering a lower energy mode must be done in software, using the
wfi and wfe instructions. Returning to EM0 and waking the CPU is initiated with
an event or interrupt to the CPU. \cite{EFM32GGManual}

\section{Motherboard}

The kit's motherboard hosts the MCU module and the prototyping board, along with
various other peripherals. Of relevance to this exercise is the Advanced Energy
Monitoring (\textbf{AEM}) system, which provides tracking of energy consumption.
There is also an LCD display, buttons and a joystick which are used to configure
the kit and display an energy consumption graph. Energy consumption can also be
tracked from a computer connected over USB. \cite{DK3750Manual}

\section{Prototyping Board and Controller}

The prototyping board provides contacts for connecting external circuits to the
dev kit. \cite{DK3750Manual} For this exercise we are connecting the controller
to two sets of GPIO pins on the board, each set consisting of 8 pins. One set of
pins is used for input from the controller's 8 buttons, while the other set is
used for setting 8 LED lights on the controller board.

\section{Software}

As mentioned two sets of software were primarily used to develop the program for
the dev kit. The GNU toolchain is a set of tools for building executable code
from source code. The toolchain also provided a debugger that was used in
conjunction with a debugging server from the energyAware software suite to debug
the program on the MCU. Other tools from the energyAware software suite was used
to upload the program to the MCU's flash memory, and to read energy consumption
information from the AEM. \cite{TDT4528Compendium}

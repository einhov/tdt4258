\chapter{Introduction}

``Let there be light'', he said, and through lines of Assembly code the LED
lights started to flicker at one hertz each. The task at hand was to write
assembly code, which would be flashed onto the flash memory on the
microcontroller unit.

Manipulating the Assembly code is consisting of changing values at a specific
register in order to turn on lights. The code may also be manipulated and
changed based on input from a specific gamedpad supplied for the task. The
gamepad includes eight buttons and the same amount of LED lights which is used
for both the inputs and also the outputs in this exercise.

The exercise only specified that we should learn startup code for ARM
processors and how to program GPIO. This would be used to detect button presses
on the gamepad and to turn on or off LED lights. Since we did not get any other
tasks for this exercise, we decided to get creative and write an Assembly code
which would accept a specific sequence of inputs and blink all the lights if
done correctly. Until this has happened every odd-numbered light has an on
state and every even-numbered light has an off state, and every second all
lights are flipped to the other state.

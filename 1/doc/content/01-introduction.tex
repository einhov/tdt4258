\chapter{Introduction}

For this exercise, we were supplied with a Development Kit, consisting of a microcontroller unit, a motherboard and a prototyping board. We were also supplied with a custom gamepad and manuals for all of these parts. In addition to this, we got a thorough explanation on how to connect the computer to the board and flash the compiled Assembly code to the flash memory of the microcontroller unit.

The Assembly code is manipulated by changing values at a specific
register in order to turn on lights. Events may also happen based on input from the provided gamepad. The
gamepad includes eight buttons and the same amount of LED lights which is used respectively
for the inputs and the outputs in this exercise.

The exercise specified that we should learn startup code for ARM processors
and how to program GPIO. This would be used to detect button presses on the
gamepad and to turn on or off LED lights. Since we did not get any other tasks
for this exercise, we decided to get creative and write an Assembly code which
would accept a specific sequence of inputs and blink all the lights if done
correctly. Until this has happened every odd-numbered light has an on state and
every even-numbered light has an off state, and every second all lights are
flipped to the other state.

Another point we decided to check was whether or not different alternatives would yield different results in energy consumptions. We focused on using timers and their ability to accurately flip the lights and their preferred usage over loops, where timers do not require the CPU to be active at all times.
\chapter{Conclusion}

In this exercise we wrote a program for a development kit for an MCU designed
for embedded applications. The program was implemented in ARM Assembly and
compiled into machine code. This machine code was then flashed onto the flash
memory of the MCU. The code included both a reset handler with instructions for
the CPU as it starts and instructions for how the CPU should act in response to
certain exceptions.

Our main task was to create a program which would react to interactions with
buttons and provide some sort of result by manipulating an array of LED lights.
To start with we implemented a polling loop that repeatedly applied input values
from the buttons over the output for the lights. In search for more efficient,
and more interesting, solutions we replicated the functionality with interrupts
while the CPU mainly slept.

We decided to get creative and set a goal beyond a direct mapping between button
presses and light states. We ended up with a program which uses a timer driven
by a clock to rotate the LED lights in a specific pattern and at a specific
interval. We also made the program recognise a specific sequence of inputs which
would trigger a change in the light's pattern.

During the exercise we learnt that there are multiple ways to achieve our task
with the tools the MCU provides. From observing energy consumption levels as we
experimented with the different tools we realised that the different tools had
noticable differences in energy requirements, and often vastly so. It became
clear that to maximise energy efficiency we had to choose carefully what tools
to apply to solve the task, and configure them appropriately. In the end we had
a program using around 1.7uA of energy while rotating lights and waiting for
button presses. That is, around \(1/2000\)th of the initial implementation. It
is clear that careful design of software for this MCU may result in large energy
savings. We also observed that the LEDs we powered pulled up to over 100mA, so
in the scope of the program some of the energy optimisations on the MCU were
felt to be rather insignificant. We do, however, appreciate that they may be
valuable in optimising applications with less energy demanding external
components.

\chapter{Introduction}

For this third exercise in TDT4258 we were tasked with implementing a game on
the EFM32GG-DK3750 development board. The game was to run on a Linux system on
the development board, drawing visuals on the development board's graphical
display and reading input from the gamepad over GPIO. The game was supposed to
run as a user application, getting gamepad data from a kernel mode device driver
we were also tasked to implement.

For our game we chose to implement a version of a board game known as
\emph{Concentration}. In concentration a grid of tiles are on a table face down.
Each tile has a corresponding match. The player flips pairs of tiles. If the
tiles match, that pair is unlocked and will remain face up. When every pair has
been unlocked, the player has won the game. The motif of the tiles in our game
are Shiba Inu dogs. We have therefore named our game Shiba Inu Match.

To start the game successfully, the gamepad driver must first be loaded. This is
done by issuing \texttt{modprobe driver-gamepad} in the shell on the device.
After the module has loaded, the game is started by issuing \texttt{game}.

The controls for the game are simple. The leftmost buttons on the controller
moves a cursor around. On the rightmost buttons, the left button selects a tile.
The game can also be turned off with the up button of the rightmost buttons,
returning the Linux system to the shell. The game has an intro image, which is
progressed through by clicking any button on the controller, except for the
turn off button. Similarily, after winning the game there is a victory
animation, which will send the player back to the intro image upon any button
press, except for the turn off button.

The game can also be started directly in the game mode or the victory animation,
by issuing \texttt{game 2} or \texttt{game 3} respectively. This is a nice
shortcut to watch the victory animation without having to finished a game of
Shiba Inu Match.

\chapter{Methodology}

\section{PTXdist and Linux}

\subsection{The build system}

Using the PTXdist build system proved to be a quite simple process. All commands
given to the build system are invoked through the \texttt{ptxdist} executable.
Before we could invoke the build system on the support files, we had to
configure PTXdist to use the correct configurations. This was done by invoking
the \texttt{ptxdist select <ptxconfig>}, \texttt{ptxdist platform
<platformconfig>} and \texttt{ptxdist toolchain <path to toolchain>} commands
with their respective arguments. After configuring PTXdist, to get the initial
support files compiled and flashed all we found necessary to do was to invoke
\texttt{ptxdist go}, \texttt{ptxdist images} and \texttt{ptxdist test
flash-all}. These commands respectively built and installed all selected
packages that had been configured, generated images ready to be flashed onto the
development board, and flashed all the images to the board.

The initial build and flash took a significant amount of time, so luckily the
entire build is not required for changes to the driver or the game. Instead,
individual packages could be updates by cleaning the previously built package
with \texttt{ptxdist clean <package>}, compile it with \texttt{ptxdist compile
<package>}, install it onto the root filesystem with \texttt{ptxdist
targetinstall <package>} and finally flash only the root filesystem to the
development board with \texttt{ptxdist test flash-romfs}. This made testing
changes to our code iteratively a more pleasant process it would have been if
rebuilding and reflashing everything had been required.

\subsection{Interacting with Linux}

With Linux flashed onto the development board and a terminal emulator
communicating over the serial connection running on our development system, we
got the kernel boot log and eventually got to a shell prompt. From there we
started the game executable by invoking \texttt{game} and loaded the gamepad
kernel module with \texttt{modprobe driver-gamepad}.

\section{Implementing the Device Driver}

While the ultimate goal of the device driver, namely accessing hardware
registers to read gamepad data, is rather simple, the matter is complicated a
bit by the necessity to work with the abstraction layers of the operating
system. The driver module must implement interfaces imposed by the kernel in
order to expose its own functionality to the rest of the system.

\subsection{Init, Exit and Boilerplate}

The main entry and exit points for the module are the gamepad\_init and
gamepad\_cleanup functions. These two functions are exported from the module
with the module\_init and module\_exit macros and are called when the module is
loaded and unloaded, respectively. In gamepad\_init we set up the driver,
allocate resources and configure the GPIO hardware. In gamepad\_cleanup we free
our allocated resources and make sure the system will be in a valid state after
the module has been unloaded. The MODULE\_DESCRIPTION and MODULE\_LICENSE macros
are also used to encode a description and code licensing information in the
module.

\subsection{Adding a Device}

Information about our device is stored in a few global variables in the module.
\texttt{dev\_t dev} will hold the \emph{device numbers} for our device.  A
second variable \texttt{struct cdev cdev} will hold information about our
character device. A third variable \texttt{struct class *cl} will contain a
pointer to a \emph{class structure}. Fourthly, a variable \texttt{struct
file\_operations gamepad\_fops} holds function pointers to our character
device's file operation functions.

The first step of adding a device to the system, is to have a pair of numbers
allocated for our new device. This is done by calling:

\begin{lstlisting}
	int alloc_chrdev_region(dev_t *dev, unsigned int firstminor,
		unsigned int count, char *name):
\end{lstlisting}

This function allocates a range of \texttt{count} minor device numbers from a
base of \texttt{firstminor} for a single dynamically allocated major device
number, storing the major number and first minor number in dev. The character
device structure \texttt{cdev} is together with a pointer to
\texttt{gamepad\_fops} sent as arguments to:

\begin{lstlisting}
	void cdev_init(struct cdev *cdev,
		struct file_operations *fops);
\end{lstlisting}

This initialises our character device structure, registering it with the correct
file operations.

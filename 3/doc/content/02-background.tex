\chapter{Background and Theory}

\section{Linux and \textmu Clinux}

In exercise 1 and 2 we programmed the development board in an unmanaged
environment, also referred to as \emph{on bare metal}. While coding on bare
metal makes it simple to quickly get code running and gives direct access to
hardware, it also puts the burden of resource management and hardware
configuration on the application programmer. The programmer must make sure to
either initialise the hardware and keep track of resources inside the
application, or to link with and call existing code that performs these tasks.
For many applications the power gained from having direct access to the hardware
is likely to be unnecessary, the required code would be similar boilerplate
replicated in multiple applications and potentially prone for bugs, and the
application programmer must expend mental effort that could have gone towards
development of the application itself. The resulting application would also be
specialised to run on the specific platform it initially targetted and
supporting additional platforms would rapidly increase the complexity of its
complexity. An alternative to programming on bare metal is to develop
applications for a managed system that does hardware initialisation and
configuration, and manages resources for the the application programmer. Such a
system is called an \emph{Operating System} (\textbf{OS}).

For this exercise the application was developped to run on an OS based on the
\emph{Linux kernel}. The Linux kernel is an \emph{UNIX-like} kernel, which means
that its interfaces to user applications mimic ones from a family of OSes known
as \emph{UNIX}. Of relevance to this exercise is that, due to Linux' UNIX-like
nature, it closely follows the \emph{POSIX standard}, which gives us a formally
defined API to code for. We will get back to POSIX and introduce the \emph{C
Standard Library} in the next section. The Linux kernel is often paired with a
set of applications, such as a shell and base tools to operate the system, and a
C library to form a complete OS. On desktop PCs, software from the GNU
collection is normally used. On embedded systems, such as in our exercise, the
software in the GNU collection is usually unsuitable due to its size and
performance footprint. The system we programmed for is instead using the
\emph{BusyBox} utilities and \emph{\textmu Clibc} library, which we will get
back to shortly.

The standard Linux kernel is designed to run on processors with a \emph{Memory
Management Unit} (\textbf{MMU}). The MMU is a unit of a CPU responsible for
translating and validating memory accesses. With an MMU a system can, amongst
other functionalities, map \emph{virtual memory addresses} to \emph{physical
memory addresses}, allowing programs to address memory independently of its
physical location, or perform access control to memory based on a program's
priveleges. As the EFM32GG microcontroller on the EFM32GG-DK3750 development
board we are using for the exercise does not have an MMU, we are using a port of
Linux called \textmu Clinux, which is a port of the Linux kernel patched to
function on microcontrollers lacking MMUs.

\section{C Standard Library and POSIX}

\section{BusyBox}

\section{ptxdist}

\section{Framebuffer}

\section{Device Drivers}
